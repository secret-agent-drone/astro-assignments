\documentclass[a4paper]{article} % A4 paper and 11pt font size

\usepackage{braket}
\usepackage{amsmath}
\usepackage{amssymb}
\usepackage{bm}
\usepackage[utf8]{inputenc}
\usepackage{verbatim}
\usepackage{tikz}
\usepackage{pgfplots}
\usepackage{siunitx}

\usepackage{fancyhdr}
\usepackage{pdflscape}
\usepackage{bm}
\usepackage{enumitem}

\usepackage[a4paper]{geometry}
\usepackage{framed}
\usepackage{gensymb}


%for side-by-side figures
\usepackage{graphicx}
\usepackage{caption}
\usepackage{subcaption}

\setlength{\parindent}{2em}
\setlength{\parskip}{1em}
\renewcommand{\baselinestretch}{1.2}


\newgeometry{bottom=4cm}

\begin{comment}
 \geometry{
 a4paper,
 total={210mm,297mm},
 left=40mm,
 right=40mm,
 top=20mm,
 bottom=20mm,
 }
 \end{comment}


%Bigger fractions (no shrinking text to fit)
%\let\oldfrac\frac
%\let\frac\dfrac

%----------------------------------------------------------------------------------------
%	TITLE SECTION
%----------------------------------------------------------------------------------------
\setlength\parindent{0pt} % Removes all indentation from paragraphs - comment this line for an assignment with lots of text


\pagenumbering{arabic}
\begin{document}
\pagestyle{empty}

\newcommand{\HRule}{\rule{\linewidth}{0.5mm}}

\begin{titlepage}

    \begin{center}
        \textsc{}\\[3cm]

        \HRule \\[0.5cm]
        \Huge \textbf{PHYC30019 Astrophysics}\\[0.5cm]
        \huge \textbf{Project 1:} Something\\[0.5cm] 
        \HRule \\[1.5cm]

        \begin{minipage}{0.5\textwidth}
        \begin{center}

		\vspace{3cm}
        \large By \\[0.75cm]
        \begin{tabular}{rl}
        \Large Braden &\Large \textsc{Moore} \\[0.1cm]
        \Large Jude & \Large \textsc{Prezens} \\    
		\end{tabular}  
		\\[1cm]
        \normalsize \normalfont 
        The University of Melbourne \\[2cm]

        \end{center}
        \end{minipage}

        \vfill

        \large \today
    \end{center}

\newpage
\end{titlepage}
%----------------------------------------------------------------------------------------
\begin{comment}
\pagestyle{fancy}
\pagenumbering{gobble}
\tableofcontents
\newpage
\end{comment}
\pagenumbering{arabic}
\rfoot{\textsc{PHYC30012 Astrophysics}}

\pagestyle{fancy}
\setcounter{page}{1}
\section{Order-of-magnitude estimate 1}
\begin{framed}
Are there more grains of sand on the beaches of Earth or more stars in the Milky Way?
\end{framed}

\subsection{Grains of sand}
We begin by estimating the number of beaches on Earth.
\begin{align*}
\text{no. of continents}&=7\sim 10\\
\text{no. of beaches/continent}&\sim 10,000\\
\Rightarrow& 100,000 \text{ beaches on Earth}
\intertext{We then make an estimate of the average volume of a beach covered by sand.}
\text{average length of beach}&=\SI{1000}{m}\\
\text{average width\footnote{Width is taken as the distance from the average water level to the grassland/non-sandy area beyond the beach} of beach}&=\SI{10}{m}\\
\text{average depth\footnote{Depth is taken as the vertical distance from surface sand to the first point below the beach surface where there is no longer sand.} of beach}&=\SI{2}{m}\\
\Rightarrow&=\SI{20,000}{m^3}\text{ of sand per beach}
\intertext{Next we estimate the number of grains of sand in a $\SI{1}{cm^3}$ box to be $\sim 1000$.}
\Rightarrow &1000\text{ grains of sand/} \si{cm^3}\\
\Rightarrow &10^3\times 10^6 \text{ grains of sand/}\si{m^3}
\intertext{We can now estimate the number of grains of sand on the beaches of Earth as}
\text{number of grains of sand}&\sim 100,000\text{ beaches} \times \SI{20,000}{m^3}/\text{beach}\times 10^9\text{ grains/}\si{m^3}\\
&\sim 2\times 10^5 \times 10^5 \times 10^9 \text{ grains}\\
&\sim 2\times 10^{19}\text{ grains}
\end{align*}

\subsection{Stars}
We begin by determining a rough measure of volume for the Milky Way.
\begin{align*}
\text{diameter of Milky Way}&\sim \SI{100,000}{ly}\\
\text{thickness of Milky Way}&\sim \SI{2,000}{ly}\\
\Rightarrow \text{volume}&=\pi\left(\frac{\SI{100,000}{ly}}{2}\right)^2 \times \SI{2,000}{ly}\\
&\sim \frac{6}{4}\times 10^3\left(10^5\right)^2 \si{(ly)^3}\\
&\sim 1.5\times \SI{d13}{(ly)^3} 
\intertext{We then estimate the average distance between stars. To do this, we take the distance from the Sun to its nearest neighbour as $\sim\SI{4}{ly}$. Using this distance we then suppose that each star exists within a ``starbox'' of dimensions $2\times 2\times \SI{2}{(ly)^3}$.}
\text{Volume per star}&\sim \SI{8}{(ly)^3}/\text{star}\\
\intertext{Now we can estimate the number of stars as:}
\text{Number of stars in the Milky Way}&= \frac{\text{volume of Milky Way}}{\text{volume/star}}\\
&\sim \frac{1.5\times 10^{13}}{8}\\
&\sim\frac{10^{13}}{10}\\
&= \SI{d12}{stars}
\end{align*}

\subsection{Conclusion}
From the above, we claim that
\begin{align*}
\text{Number of grains of sand on the beaches of Earth}&\sim 10^{19}\\
\text{Number of stars in the Milky Way}&\sim 10^{12}
\end{align*}

From the rough order-of-magnitude calculations performed, we deduce \emph{there are more grains of sand on the beachs of Earth than there are stars in the Milky Way}. Even though we have made many assumptions as well as liberal rounding of numbers, it is unlikely that these values are more than an order of magnitude greater or smaller than quoted. 

It is validating to note that the common answer for number of stars in the Milky Way\footnote{See http://asd.gsfc.nasa.gov/blueshift/index.php/2015/07/22/how-many-stars-in-the-milky-way/} is quoted as $\sim 10^{11}$.


\section{Research task 1}
\begin{framed}
What is the lowest mass for a star? 
\end{framed}

THe lowest mass for a star is the mass of a brown dwarf, with mass
\begin{equation}
M_{\text{BD}}\sim (13\to 65) M_{\text{Jup}}
\end{equation}
where $M_{\text{Jup}}$ is the mass of Jupiter. In terms of solar mass, this gives
\begin{equation}
M_{\text{BD}} < 0.07M_{\odot}
\end{equation}

\begin{framed}
What Physics determines this mass?
\end{framed}

Minimum mass is determined by the Jeans mass
\begin{equation}
M_J=\frac{3kT}{G\mu m}r
\end{equation}


\begin{framed}
Will compact objects form below this mass - and if so what are they?
\end{framed}

\begin{framed}
Can we observe them?
\end{framed}


\section{Calculation 1}
\begin{framed}
What is the most massive star that can form? State the physics that you have used the derive this estimate. How could a star form that might avoid this limit?
\end{framed}

\section{Calculation 2}
\begin{framed}
Suppose the gas cloud of $10^3 M_{\odot}$ collapses under gravity and forms stars. Under simple assumptions, what is the largest star that will form? You may assume the Salpeter IMF which has the form: $dN/dm\propto m^{-2.35}$.
\end{framed}

\section{Calculation 3}
\begin{framed}
Once a gas cloud reaches an overdensity of $\sim 1$, it will colla[se in free fall. Determine the time scale of collapse of a gas cloud of mass $M$. Work out how this scales. What will stop our gas cloud collapsing into a black hole?
\end{framed}




\end{document}







